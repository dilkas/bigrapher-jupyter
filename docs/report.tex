\documentclass{article}
\usepackage[UKenglish]{babel}
\usepackage[UKenglish]{isodate}
\usepackage{url}
\usepackage{listings}

\author{Paulius Dilkas}
\title{Towards User-Friendly Bigraphs}

\begin{document}
\maketitle

\begin{abstract}
  We adapt an OCaml Jupyter kernel to support BigraphER.
\end{abstract}

\section{Introduction}

\subsection{Jupyter}

Project Jupyter \cite{website:jupyter}, notebook \cite{website:docs}, kernel,
OCaml kernel\footnote{\url{https://github.com/akabe/ocaml-jupyter}}, cell.

\subsection{Bigraphs and BigraphER}

Bigraphs, BigraphER \cite{Sevegnani2016}

begin-end block, stochastic and non-stochastic reaction rules

Goals: convenience of use, supporting a similar workflow to how Jupyter works
with Python, not surprising the user with unexpected behaviour.

\section{Assumptions}

\begin{itemize}
\item Keywords \texttt{big}, \texttt{react}, \texttt{begin} start the beginning
  of line.
\item Single spaces or no spaces.
\end{itemize}

\section{Features}

% TODO: examples of error handling
\begin{itemize}
\item Variables defined in one cell persist to the next (unless the cell
  contains a begin-end block or fails to run). This is done in the same order as
  the cells are run, including running the same cell multiple times.
\item If a cell doesn't have a begin-end block, a dummy one is added (but does
  not persist).
\item The output of each cell corresponds to everything defined in that cell
  (bigraphs and reaction rules): name first, then diagrams.
\item Reaction rules are visualised in an HTML table, connecting the diagrams
  side-by-side.
\item All images are saved in a directory \texttt{jupyter-images}, with a
  subdirectory for each cell. If a subdirectory doesn't exist, it is created.
  Otherwise, its contents are cleared before the new images are added. Stale
  directories are not deleted.
\item Both stochastic and non-stochastic rules are supported, as long as they
  are not in the same cell.
\item OCaml code can be run if the first line is \texttt{\%ocaml}.

\item BigraphER API can be called from an OCaml cell. That requires two lines in
  \texttt{.ocamlinit} to load stuff.
\item Auto-complete and integrated documentation for BigraphER OCaml API (and
  other OCaml code) using Merlin.
\end{itemize}

\section{Implementation}

\begin{itemize}
\item Separate buffers for OCaml and BigraphER code.
\item Tests with good code coverage.
\item BigraphER's version along with OCaml version in the kernel's name.
\item Dummies are implemented with random words that are not defined anywhere else.
\item Code from buffer is written to a file, BigraphER is run to generate
  images, file is deleted. BigraphER's output is captured, but not used, since
  combining multiple cells results in warnings about multiple definitions.
\item Directory permissions are $700$.
\item \texttt{react} keyword is ignored if it's not at the start of a line.
\end{itemize}

When evaluating a cell, we look for a begin-end block in a top-down
fashion (but ignoring the buffer).
\begin{itemize}
\item In case the model turns out to be non-stochastic (starts with
  \texttt{begin brs}), no additional work needs to be done.
\item If we find a stochastic (\texttt{sbrs}) model, we must remove all
  non-stochastic reaction rules from both the buffer and the cell.
\item If such a block is not found, a dummy \texttt{brs} block is generated
  for the purpose of running BigraphER and generating the required images. The
  dummy \texttt{begin}-\texttt{end} block has the form:
  \begin{lstlisting}
ctrl C = 0;
big b = C.1;
react r = b --> b;
begin brs
  init b;
  rules = [{r}];
  preds = {b};
end
\end{lstlisting}
\texttt{C}, \texttt{b}, and \texttt{r} are randomly generated words,
distinct from each other and other variables defined in the cell or buffer.
The words are generated by adding random letters until the constructed
string becomes unique. For simplicity of implementation, \texttt{C} always
starts with a capital letter C.
\end{itemize}

\section{Conclusion}

\bibliographystyle{plain}
\bibliography{references}
\end{document}